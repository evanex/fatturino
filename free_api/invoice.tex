\documentclass{invoice} % Use the custom invoice class (invoice.cls)
\usepackage{eurosym}
\usepackage{tabularx}
\usepackage{fancyhdr}
\usepackage{setspace}
\def \tab {\hspace*{3ex}} % Define \tab to create some horizontal white space
\begin{document}
\begin{center}
\Huge\bf $ragione_sociale$ % Company providing the invoice
\end{center}
\begin{tabularx}{\textwidth}{l X   r}
$indirizzo$ & &  $citta_cap$   \\ [0.1ex]
P.IVA $partita_iva$ & &  C.FISC $codice_fiscale$ \\ [0.1ex]
\end{tabularx}

\bigskip
\bigskip

\begin{tabularx}{\textwidth}{l X l  r}

{\bf Spett.le:}  & &   {\bf Data:}  & $data_fatt$   \\
\tab  $dest_ragione_sociale$  & &  {\bf Numero fattura:}   & $num_fatt$ \\ 
\tab $dest_idirizzo$  & &  & \\
\tab $dest_citta_cap$  & &   & \\
\tab P.IVA $dest_partita_iva$  & &  &  \\ 
\tab C.FISC $dest_codice_fiscale$ & &  &  \\ 
\end{tabularx}

\bigskip
\bigskip
\feetype{$descrizione$} % Fee category description
\bigskip
\begin{center}
 \begin{tabularx}{\textwidth}{X  c r }

$RIGHE_FATTURA$
\end{tabularx}
\end{center}


$PROFORMA_TEXT$

\bigskip
\vspace*{\fill}
La Risoluzione n.132/E del 28 maggio 1997 del M.d.F. precisa che le informazioni stampate su fatture
possono essere trasmesse elettronicamente al cliente senza provvedere all'invio in forma cartacea
del documento; il cliente, a sua volta può provvedere direttamente alla stampa o alla registrazione
dei documenti elettronici mediante la memorizzazione dei dati.

%----------------------------------------------------------------------------------------

\end{document}